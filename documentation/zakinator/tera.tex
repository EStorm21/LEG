\section{Tera}

\subsection{Brief Overview}

During the Fall 2014 semester, we managed to get the main testing server Tera up and running again. This included installing new SSDs, reinstalling the operating system, setting up a RAID array, and all-around stability improvements. Tera is now operational and provides the correct functionality for all Clay-Wolkin research teams, as well as members of both Bassman's and Spencer's research teams. Current admins are Zakkai Davidson, Daniel Johnson, Api Sharma, and Avi Thaker. If anything is broken, check with one of them!

\subsection{Basic Configuration}

We performed a clean install of Centos 6.5 (which has since updated to 6.6), so any old library files are no longer available and will need to be reinstalled as needed. All user files from the older system are still present in the \texttt{/home} directory, and can be accessed by the users that created them. Other files from the older version of tera can be found under \texttt{/hdd}.

\subsection{Common Tasks}

\textbf{Creating new users:}

\noindent \texttt{/usr/sbin/adduser -g users [username]}



\textbf{Installing new packages:}

\noindent Use yum to install everything. Helping features include \texttt{yum search} to find a specific package and \texttt{yum whatprovides} to find what package provides a spefic library file. Note: this always needs to be run as root.



\subsection{Known problems (as of 29 Apr 2015)}

\begin{enumerate}
\item On restart, network DNS settings often reset to default. \textbf{Fix:} copy the contents of \texttt{/hdd/etc/resolv.conf} (copied below for reference) into \texttt{/etc/resolv.conf}. This is what the file should look like to function correctly:

\begin{verbatim}
search localdomain
nameserver 134.173.53.8
nameserver 134.173.254.23
\end{verbatim}

\item During VLSI checkoffs, tera when into kernel panick twice. It might be related to the heavy load during that period, but it is definitely something to look more into. Tim Bucheim suggested that we should run memcheck because it seems like a RAM issue.
\end{enumerate}

