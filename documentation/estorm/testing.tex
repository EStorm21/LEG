% TODO
\section{Name of thing that I made}

\subsection{Brief Overview}

% TODO

\subsection{Important Diagrams}

% TODO - include images or give path to images in git

\subsection{All Relevant Files and Brief Descriptions}

% TODO

\subsection{Analysis of Confusing Sections of Code}

% TODO

\subsection{Running Tests}
%TODO how to create and compile tests
%TODO running and debugging Linux, locally and on Tera

Running tests requires modification of the default \texttt{autoSim0.tcl} 
file in the project root directory (assumed in this document to be \texttt{/LEG} )
and the \texttt{dmem.sv} file in \texttt{LEG/leg\_pipelined/dmem.sv}. 
Note that the path after \texttt{/LEG} is shown, but absolute paths are 
required for all files and directories.
\\\\
\large \textbf{Modifying \texttt{autoSim0.tcl}}

\begin{enumerate}

\item Create an empty ModelSim project in your favorite directory.

\item Create an unversioned copy of \texttt{autoSim0.tcl} for your own use.

\item Modify the entries in the copy to point to the correct locations:

	\begin{itemize}

	\item \texttt{project} should point to your ModelSim project.

	\item On your own computer, \texttt{vlog} should end in \texttt{LEG/leg\_pipelined/*.sv}

	\item \texttt{testPath} should point to \texttt{LEG/tests}

	\item \texttt{testList} should point to \texttt{LEG/tests/tests.list}

	\item In the test for loop, change the location of simTest.dat to a directory 
	of your choice, e.g. \texttt{LEG/tests/simTest.dat}

	\end{itemize}

\end{enumerate}

\large \textbf{Modifying \texttt{dmem.sv}}

Comment out the other \texttt{\$readmemh} lines, and add your own with 
the same path to the \texttt{simTest.dat} file. Each time you pull from 
the repository and want to run tests you will need to uncomment 
the correct line in this file.

\bigskip
\textbf{Running tests}

\begin{enumerate}
\item Uncomment the test you want to run in \texttt{LEG/tests/tests.list}.

\item Using Cygwin, navigate to the folder containing your \texttt{autosim.tcl} file. 

\item run the command \texttt{vsim -c -do autosim.tcl}

\item You can open the ModelSim project during or after a run of tests,
and the waveforms will be available.
\end{enumerate}
