% TODO
\section{Data Cache}
This section describes the structure of the data cache and its role in the memory system.

\subsection{Brief Overview}

The data cache is a writeback, 2 way associative cache. 
The cache uses physical tagging and virtual indexing, so the number of bytes in each way is limited to the size of a tiny translation page (1024 bytes). 
The default cache size is $64 \text{ lines} \cdot 4 \text{ words/line} \cdot 4 \text{ bytes/word} = 1024 \text{ bytes}$. 
The number of lines in a way and block size are parameterized, and the replacement policy implemented is Least Recently Used (LRU).
The number of ways is not parameterized, because the Least Recently Used Logic becomes more complex as the number of ways increases.

\subsection{Important Diagrams}


% TODO - include images or give path to images in git

\subsection{All Relevant Files and Brief Descriptions}

Table \ref{table:drel} shows the files that are used by the data cache and Table \ref{table:drel} lists the top level inputs and outputs.

\begin{tabular}{|l|p{80mm}|}
\hline File  & Description \\ 
\hline  data\_writeback\_associative\_cache.sv & Top level D\$ module \\ 
\hline  data\_writeback\_controller.sv & Controller logic for the D\$.
Contains the primary state machine described in section \ref{sec:dstate}\\ 
\hline 
\label{table:drel}
\end{tabular} 

\begin{tabular}{|l|l|l|l|}
\hline Port & Description & Input/Output \\ 
\hline  &  &  &  \\ 
\hline  &  &  &  \\ 
\hline 
\label{table:dio}
\end{tabular} 

\subsection{Analysis of Confusing Sections of Code}
\label{sec:dstate}

Below is an explanation of the states NEXTINSTR, WRITEBYTES, WAIT,and DWRITE.

\begin{enumerate}
	\item NEXTINSTR
	The next instr state removes the stall on the pipeline and allows the instructions to move one stage down the pipeline. If this stage did not exist, then the instruction at the data stage would remain the same and after the requested data is retrieved, the same data would be retrieved again. If the entry were uncachable, then the processor would stall forever.

	\item WRITEBYTES
	The cache enters this state when the cache is disabled, and it needs to writeback bytes of data. In this state, the data from the memory has been loaded into the cache and the proper bytes have been written to the cache memory. After this state, the data is stored back into the main memory with the proper bytes overwritten. This state is separate from DWRITE because writing a whole word to memory does not require loading the word first. This state will be removed when a bytemask is added to the bus.

	\item WAIT
	The data cache enters this state when simultaneous data and instruction stalls occur. The data cache has bus precedence, so it waits for the instruction cache to retrieve data before handling the next request. 

	\item DWRITE
	This state handles disables full word writes to main memory.
	
\end{enumerate}

\subsection{Not complete}

\begin{enumerate}
	\item Add Byte Mask to the ahb bus
	\item Fix the infinite loop bug at linux simulation time ~1772200
\end{enumerate}