\label{sec:h}
The Hazard unit detects potential conflicts between pipeline stages and stalls, flushes, or forwards values between stages to resolve them.

\subsection{Relevant Files}
\begin{tabular}{|l|p{120mm}|}
\hline \textbf{File}  & \textbf{Description} \\ 
\hline hazard.sv & Generates flush, stall, and forward signals for the datapath and controller based on instruction types and processor state. \\
\hline eqcmp.sv & Compares register numbers at different pipeline stages to inform forward signals. \\
\hline 
\end{tabular} 

\subsection{Stalls}
All pipeline stages can be stalled, but the Execute, Memory, and Decode stages are stalled only because of data and instruction cache behavior.
The Fetch stage is additionally stalled during micro ops, coprocessor and CPSR operations, PC writes.
The Decode stage can be stalled for micro ops and by the exception handler.

Both Fetch and Decode are stalled during a read after write (RAW) hazard when a register is loaded from memory. 
Other RAW hazards are handled by forwarding.

\subsection{Flushes}
Various pipeline stages are flushed in the same circumstances as they are stalled. 
This behavior kills writeback signals during stalls so actions are not performed multiple times.
Additional flushes take place whenever a branch is taken since LEG does not implement branch prediction.

\subsection{Forwarding}\label{sec:fwd}
Read after write (RAW) hazards occur when a register that is written by one instruction is read by a later instruction before the change has propagated through to the Writeback stage.
Most RAW hazards can be solved by forwarding values from the Memory or Writeback stages to the Execute stage rather than stalling.
These hazards are detected by comparing the register number being used as an operand in each stage and then selecting the appropriate value to forward. 
This forwarding is gated by the corresponding stage's register writeback signal so values from unexecuted or flushed instructions are not used.