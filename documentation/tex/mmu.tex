\section{Memory Management Unit}
The memory management unit (MMU) supports most of the memory functions described in ARMv5.
The functionality implemented is fully described in section \ref{sec:mmufunct}
The MMU supports virtual address translation using Sections and Large, Small, and Tiny pages.

\subsection{Brief Overview}

The MMU consists of three main modules: the translation lookaside buffer (TLB), translation walk hardware (twh.sv), and the translation fault hardware (tfh.sv). 

\subsection{Important Diagrams}

% TODO - include diagram from the paper

\subsection{All Relevant Files and Brief Descriptions}

Table \ref{table:mmurel} shows the files that are used by the MMU and Table \ref{table:mmuio} lists the top level inputs and outputs.

	\begin{table}
	\label{table:irel}
	\begin{tabular}{|l|p{70mm}|}
	\hline File  & Description \\ 
	\hline  instr\_cache.sv & Top level I\$ module \\ 
	\hline  instr\_cache\_controller.sv & Controller logic for the I\$.
	Contains the primary state machine described in section \ref{sec:istate} \\ 
	\hline  data\_writeback\_associative\_memory.sv & 
	Memory module containing the both cache ways, the LRU memory, and way selection mux's.
	This module is used in both the instruction and data caches.
	The instruction cache fixes the dirty and clean inputs to zero, because it is a read only cache.\\ 
	\hline  data\_writeback\_associative\_cache\_way.sv & 
	Contains the memory associated with one cache way. This includes four words per line along with the valid, and tag bits. \\ 
	\hline  word\_memory.sv & Byte addressable word memory  \\
	\hline
	\end{tabular} 
	\caption{Instruction cache files}
	\end{table}

	\begin{table}
	\label{table:instrio}
	\begin{tabular}{|l|p{85mm}|l|}
	\hline Port & Description & I/O \\ 
	\hline clk & Clock input &  I \\ 
	\hline reset & Global reset signal &  I \\ 
	\hline uOpStallD & Micro-Op stall signal. This signal is used to prevent repeated memory assesses when the pipeline is stalled &  I \\ 
	\hline A[31:0] & Virtual address from the leg datapath & I \\
	\hline CP15en & Enable signal from the coprocessor & I \\
	\hline AddrOp & Indicates coprocessor is invalidating using a virtual address instead of a set index. When high, only invalidate on a hit in the instruction cache.& I \\
	\hline InvAllMid & When InvAllMid and Inv are high, invalidate all lines from cache. This signal is driven by the coprocessor & I \\
	\hline Inv & Invalidate signal from the coprocessor & I \\
	% \hline CurrCBit & Cachable bit for the current TLB entry & I \\
	\hline PhysTag[tbits-1:0] & Physical Tag from the TLB & I \\
	\hline PAReadyF & Indicates TLB entry is valid for the instruction cache. The instruction cache uses the physical address and control bits from the TLB & I \\
	\hline FSel & Indicates instruction cache has control of the AHB Bus & I \\
	\hline BusReady & AHB Ready signal & I \\
	\hline HRData[31:0] & Data from the AHB Bus & I \\
	\hline RD[31:0] & Data output from the cache. This is the same as HWData. & O \\
	\hline IStall & IStall signal from the instruction cache controller & O \\
	\hline HRequestF & Request AHB control & O \\
	\hline HAddrF[31:0] & AHB address & O \\
	\hline RequestPA & Request a physical address from the TLB & O \\
	\hline
	\end{tabular} 
	\caption{Instruction cache I/O (instr\_cache.sv)}
	\end{table}

\subsection{Analysis of Confusing Sections of Code}

\subsection{Functionality}
\label{sec:mmufunct}

\begin{list}
\item tfh.sv: 
The fault detection has been tested using directed tests and waveform inspection, but has not been against qemu or while booting linux.
For this reason, it is disabled by default.
\end{list}

